\chapter{Discussion}

Once the results of all the four optimizations performed here have been shown, they will be briefly discussed. The four cases showed how the individuals of the populations move throughout the search space with its corresponding position in the function space. Also, some sample individuals of the population are presented in order to see how the individuals behave and how they evolve towards a solution. 

\subsubsection*{Supression of cylinder vortex oscillations}

The search space is too large and, due to time restrictions, the number of individuals simulated did not cover evenly the search domain. Also, the solution converges to two points instead of a full Pareto front. The algorithm, however, behaves as it should, because it moves the points towards the zero oscillations points, which is not reached but it is close to some solutions. 

In Figure \ref{fig:initialCyl} some individuals from the first population have been randomly chosen and the figures with its oscillations are shown. The blue line represents the sum of the pressure and viscous force in the $x$ axis while the green line is the pressure and viscous force in the $y$ axis. The horizontal axis represents the time evolution while the vertical axis is the force in Newtons. The last point is chosen as a reference to pick a number of cycles that realistically represents the oscillations suffered by the cylinder. As it can be seen, there is a great variety of amplitudes and frequencies in the search space.

In Figure \ref{fig:finalCyl} the force plots for the two points where the algorithm converges are shown. The oscillations have been damped and the amplitude is smaller than the amplitudes of the initial population which was the objective of the algorithm. Thus it can be said that, although not being a Pareto front, the solutions of the algorithm have reduced the amplitude of the oscillations. 

\subsubsection*{Inlet to diffuser geometry}

In this case, Figure \ref{fig:genForDifusser} shows not only the evolution of the individuals but also the search space where they are bounded in. In contrast with the cylinder analysis, the function space shows a neat and well-defined Pareto front. The individuals may have a large pressure ratio and a low Mach number or a large Mach number and a small pressure ratio: the trade-off between the two variables has been captured.

In Figure \ref{fig:initialDif} some randomly chosen individuals from the initial generation are chosen. It can be seen that there is a great variety of $L$ and $\theta$ possible values. The cases shown do not look as optimal, and there are shapes (as the middle one) that look even unrealistic. However, in Figure \ref{fig:finalDif} it can be seen how the individuals have evolved towards a most optimal solution. The left image shows the expected solution where the shock wave hits the cowl of the engine. However, there are plenty of other solutions that are not dominated, having higher velocities for some cases and higher pressure ratios for others. 

\subsubsection*{Airfoil design: lift maximization and drag minimization}


Given that the airfoil setup was quicker than the other cases, much more individuals have been simulated (see Figure \ref{fig:genForCLCD}). The search space is well covered and the points are spread enough to see where the optimal region is. Points eventually converge to the location of that optimal area, creating some unwanted clustering zones. This is translated in the function spaces as a well-defined Pareto front that captures the trade-off between lift and drag, given that both are related. 

In Figure \ref{fig:initialCLCD} it can be seen that there is a wide range of possible airfoils obtained through the Joukowsky transform. This case was initialized with a Sobol sampling, which is quasi-random low discrepancy initialization that tries to cover the whole search space as much as possible. This initialization will be useful for the second optimization performed to the airfoil case, given that the same initial population was used to reduce as much as possible computational time. This type of initialization covers better the search space and, while for a genetical algorithm not applied to CFD this will yields the same results after a large enough number of generations, in this case, it is very interesting to reduce the simulations as much as possible. The wide range that the airfoils in the first generation have airfoils converge to more aerodynamic shapes (see Figure \ref{fig:finalCLCD}) with a low thickness and slightly cambered. This shape enhances as much as possible the lift while affecting the less possible the drag. 


\subsubsection*{Airfoil design: lift-to-drag and area maximization}

Finally, Figure \ref{fig:genForLD} shows the search space (which is the same as the previous one) and the function space with a completely different Pareto front as the one seen in Figure \ref{fig:genForCLCD}. Although the first generation was formed by the same individuals, the selection pressure based on assigning the fitness depending on different objectives yield completely opposed results. The Pareto front is well captured and it shows the most optimal solutions for maximizing the $L/D$ ratio and the area. 

Figure \ref{fig:initialLD} show randomly chosen individuals from the initial generation with some possible airfoils (having that individuals from Figure \ref{fig:initialCLCD} also are possible individuals from the first generation for the airfoil case II). The results of the final generation (see Figure \ref{fig:finalLD}) are way different from the results of Figure \ref{fig:finalCLCD}. Here the thickness is (in general) larger than in the previous case. However, there are individuals that have a smaller thickness than other individuals from the population - but being still thicker than the individuals in Figure \ref{fig:finalCLCD}.